\documentclass[10pt]{article}
\usepackage[utf8]{inputenc}
\usepackage[left=20mm, bottom=20mm, right=20mm, top=25mm]{geometry}
\usepackage{color}
\usepackage{soul}
\usepackage[T1]{fontenc}
\usepackage{colortbl}
\usepackage{fancyhdr}
\usepackage{caption}
\usepackage{mwe}
\usepackage{listings}
\usepackage{pgf,tikz}%%%%%%%%%%%%
\usepackage{float}
\usepackage{pgfplots}
\usetikzlibrary{intersections}
\usetikzlibrary{arrows.meta}
\usepackage{mathtools}
\usepackage{fix-cm}
\usepackage{multirow}%
\usepackage{pgfplots}%
\usepackage[english]{babel}
\usepackage[shadow,color=mc1,linecolor=mc4,bordercolor=mc4,textwidth=23mm]{todonotes}
\usepackage{amsmath}

\usepackage{algorithm}  % Za prikaz algoritma
\usepackage{algpseudocode}  % Za algoritamske pseudokode

\usepackage{circuitikz}
\usetikzlibrary{circuits.logic.US}
\usetikzlibrary{arrows, shapes.gates.logic.US}
\usepackage{caption}
\usepackage{tikz}
\usetikzlibrary{shadows}
\usepackage{pgfplots}
\usepackage{sidecap}
\usepackage{verbatim}


\usepackage{graphicx}
\usepackage{eso-pic}
\usepackage{lipsum}  % Dodano za primjer teksta


\usepackage{listings}
\usepackage{xcolor}

\usepackage{enumerate}
\usepackage{graphicx}
\usepackage{listings}
\usepackage{tikz}
\usetikzlibrary{calc}

\usepackage{embedfile}
\usepackage{hyperref}

\usepackage{enumitem}

\newcounter{logo_stil}
\definecolor{boja}{RGB}{15,123,142}

\newcommand{\dispenvname}[1]
{\texttt{\color{blue}#1\color{black}}~}

\newcommand{\nazivslike}[1]
{
\vspace{2mm}
\small\stepcounter{logo_stil}\hyperterget{sl;\arabic{logo_stil}}{slika}\arabic{logo_stil}.#1
\addcontentsline{lof}{figure}{\arabic{logo_stil}\hspace{4mm}#1}

\vspace{2mm}
}

\setlength{\headheight}{25pt}

\fancypagestyle{tptp_stil}
{
\lhead{\small\sc{Aljic Eldar}}\chead{}\rhead{\includegraphics[scale=0.05, trim=0.25mm 0.25mm 0.25mm 0.25mm,clip=true]{img/logo.pdf}}
\lfoot{}\cfoot{}\rfoot{\thepage}
\renewcommand{\headrulewidth}{0.35pt}
\renewcommand{\footrulewidth}{0pt}
}

%\pagestyle%%%%%%%%%%%%%%%%%%%%%%%%%%%%%%%%%%%%%%%

\hypersetup{
colorlinks,
citecolor=black,
filecolor=black,
linkcolor=black,
urlcolor=boja,
}

\definecolor{mycolor}{RGB}{15,123,142}
\definecolor{mc1}{RGB}{245,204,204}
\definecolor{mc2}{RGB}{48,134,109}
\definecolor{mc3}{RGB}{224,234,241}
\definecolor{mc4}{RGB}{197,66,67}
\definecolor{zelena}{RGB}{0,153,0}

\newcommand{\newsize}[1]{\fontsize{9pt}{21pt}\selectfont{#1}}
\newcommand{\newtodo}[1]{\todo{\newsize{#1}}}


\lstset{
    basicstyle=\ttfamily\footnotesize,
    keywordstyle=\color{blue},
    commentstyle=\color{gray},
    stringstyle=\color{red},
    breaklines=true,
    frame=single,
    numbers=left,
    numberstyle=\tiny\color{gray},
    tabsize=2,
    captionpos=b
}



\newcommand{\siva}{\rowcolor{red!20!black!10}}  %za farbanje polja sa kombinacijom boja: 20%crvene 10%crne
\newcommand{\tamsiva}{\rowcolor{black!70}}% 70% crne boje

\newcolumntype{L}[1]{>{\centering\arraybackslash}m{#1}}%za 1. tabelu

\newcounter{rbslika}

\newcommand{\brojslike}[1]
{
  \vspace{2mm}
  \small\stepcounter{rbslika}\hypertarget{sl:\arabic{rbslika}}{Slika} \arabic{rbslika}. #1
  \addcontentsline{lof}{figure}{\arabic{rbslika} \hspace{4mm} #1}
  
  \vspace{2mm}
}

%%%%%%%%%%%%%%%%%%%%%%%%%%%%%%%%%%%%%%%%%%%%%%


\newcounter{rbtabela}  
\newcommand{\evosok}[1]  
{  
\vspace{2mm}  
\small\stepcounter{rbtabela}\hypertarget{sl:\arabic{rbtabela}}{Tabelica} \arabic{rbtabela}. #1  
\addcontentsline{lot}{table}{\arabic{rbtabela} \hspace{4mm} #1}  
\vspace{2mm}  
}
%%%%%%%%%%%%%%%%%%%%%%%%%%%%%%%%%%%%%%%%%%%%%%




\begin{document}

				% Dodjeljujemo odredjene nazive komandama
\renewcommand{\figurename}{Slika}
\renewcommand{\arraystretch}{1.4}
\renewcommand{\contentsname}{Sadrzaj}
\renewcommand{\listfigurename}{Lista slika}
\renewcommand{\listtablename}{Lista tabela}
\renewcommand{\abstractname}{Sazetak ili abstract}



\noindent Univerzitet u Tuzli \hfill  Tuzla, juni 2024. \\
\noindent Fakultet elektrotehnike \hfill Satelitske telekomunikacije \\
\noindent \hfill  \\

% postavljanje wtermark slike 
\AddToShipoutPictureBG*{
	\AtPageCenter{
		\hspace{-60mm}
		\begin{tikzpicture}[remember picture,overlay,opacity=0.07]
			\node at (current page.center) {\includegraphics[width=0.5\paperwidth, angle=0]{img/logo.pdf}};
		\end{tikzpicture}
	}
}



\begin{center}

\vspace{60mm}

\LARGE{}\textbf{Izvjestaj} \\
\vspace{5mm}
People Counting with Python
\end{center} 
\vspace{5mm}

\vfill{
	\noindent \textbf{Studenti:} \\ % \hfill Profesor:  \\
	\noindent Eldar Aljic  \\
	\noindent Benjamin Hodzic   \\ 
	\noindent Dzana Dugonjic \hfill \textbf{ Predemtni nastavnik } \\ 
	\noindent Osman Azabagic  \hfill Dr.sc. Alma Šećerbegović, docent}
\vspace{10mm}



\newpage

\begin{abstract}

\noindent Pisemo sazetak tj. abstract da vidimo da li ce mi se ova promjena desiti \\
\noindent ovo je neka nova promjena  \\
\noindent pa sada kao bude upade 

\end{abstract}


\newpage

\tableofcontents

\newpage

\listoffigures
%\listoftables





\newpage
\pagestyle{tptp_stil}

\section{Uvod}
\noindent Pisemo ovod za ovaj izvjestaj ili sta god da pravim


\section{Definicija problema}
\noindent Ovako pravimo drugu sekciju 


\begin{enumerate}
    \item \textbf{Ovo je sa: } brojevima 
    
    \item \textbf{bla bla} 
\end{enumerate}

    	\begin{itemize}
    		\item  ovo je sa tackama
		\end{itemize}
		

\begin{itemize}


    \itemsep0em % Ovo redukuje razmak između stavki u listi
    \item prvi dio:
    
		\begin{itemize}
    		\item linija unutar prva 

		\end{itemize}    
    
    	
    \item drugi dio:
    
		\begin{itemize}
    		\item Ako
		\end{itemize}      
    
    

\end{itemize}





\subsection{Na ovaj nacin pravimo subsection }



% Da se zakomentarise selektovani kod ctrl + T, a otkomentarise ctrl + U 

% Sliku ubacujemo na sljedeci nacin : 

%%\vspace{5mm}
%\begin{figure}[H]
%	\centering
%	\includegraphics[width=0.7\linewidth]{img/stk_1b.png}
%	\caption{MobileNet SSD kod 1b}
%	\label{fig:stk_1b}
%\end{figure}
%\vspace{5mm}


\subsubsection{Na ovaj nacin pravimo subsubsection}








\pagestyle{tptp_stil}
\newpage










% takodjer lijep nacin zapisivanja nista vise :)
\begin{verbatim}
frame = cv2.cvtColor(frame, cv2.COLOR_RGB2BGR)
cv2.imshow("Frame", frame)
key = cv2.waitKey(1) & 0xFF
\end{verbatim}




% na ovaj nacin mozemo da zapisemo neki kod ovo je kao python kod mozes 
% podesiti koji kod zelis da bude kao obojene komande
\begin{lstlisting}[language=Python, caption=Primjer Python koda]
fps.stop()
print("[INFO] elapsed time: {:.2f}".format(fps.elapsed()))
print("[INFO] approx. FPS: {:.2f}".format(fps.fps()))

if writer is not None:
    writer.release()

if not args.get("input", False):
    vs.stop()
else:
    vs.release()

cv2.destroyAllWindows()

d = [datetime.datetime.now()]
dts = [ts.strftime("%A %d %B %Y %I:%M:%S%p") for ts in d]
export_data = zip_longest(*[dts, empty, empty1], fillvalue='')

with open('Log.csv', 'w', newline='') as file:
    writer = csv.writer(file)
    writer.writerow(("End Time", "In", "Out"))
    writer.writerows(export_data)

\end{lstlisting}




 \end{document}
